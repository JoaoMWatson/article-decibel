\section{Discussão}

\subsection{Definição}
A unidade bel é usada para caracterizar o logaritmo decimal da razão entre duas quantidades similares de energia ou potência $P_1$ e $P_2$:\\
    $$Q_(P) = \log\frac{P_1}{P_2} B = 10 \log\frac{P_1}{P_2} dB$$\\
Para $Q_{(P)}$, por exemplo, o valor 1 B resulta quando a razão de potência é $P_1/P_2 = 10$.
$$1dB = \frac{1}{10} B$$

Portanto, o bel representa o logaritmo de uma razão entre duas grandezas de potência de 10:1, ou o logaritmo de uma proporção 
entre duas grandezas de potência-raiz de √10:1.[9] Logo, duas quantidades cujos níveis diferem em um decibel têm uma relação de potência de 101/10, 
o que corresponde aproximadamente a 1,25893, e uma amplitude (quantidade potência-raiz) cujo rácio é 101⁄20 (aproximadamente 1,12202).

\subsection{Historia}
Em seus primeiros estudos com a acústica, Alexander Graham Bell (1847 – 1922), 
inventor do telefone, entre outras coisas, percebeu que a variação de som que 
o ouvido humano pode sentir não acompanha uma escala linear.

Isso significa que se dobrarmos a amplitude de um sinal (duplicar sua tensão elétrica), 
nosso ouvido não perceberá como sendo o dobro da pressão sonora recebida, ou melhor, o dobro do volume.
Graham Bell notou que a escala que o ouvido percebe é logaritma.

Portanto, ao invés de utilizar a escala linear para representar a amplificação (ganho) ou a atenuação (perda) de um sistema, 
Graham Bell resolveu utilizar uma escala logaritma.
Graham Bell criou uma unidade de medida para esta atenuação. Esta unidade era chamada originalmente de TU (transmission unit), pelo próprio Graham Bell.

Mas em 1929, após a sua morte, os engenheiros do Bell Telephone Laboratory resolveram homenagear seu fundador, 
dando o nome de Bel (símbolo B) a esta unidade de medida.

Por definição do Bell Labs, 1 Bel é igual a atenuação em um sinal de áudio em uma milha (1,61 km) de cabo telefônico.

\section{Tabela}
\begin{table}[hb]
    \centering
    \begin{tabular}{|c|c|c|}
         \hline
         dB & Rácio de Potências & Rácio de Amplitude \\
         \hline
         100 & 10 000 000 000 & 100 000 \\
         \hline
         90 & 1 000 000 000 & 31 623 \\
         \hline
         80 & 100 000 000 & 10 000 \\
         \hline
         70 & 10 000 000 & 3 162 \\
         \hline
         60 & 1 000 000 & 1 000 \\
         \hline
         50 & 100 000 & 316.2 \\
         \hline
         40 & 10 000 & 100 \\
         \hline
         30 & 1 000 & 31.62 \\
         \hline
         20 & 100 & 10 \\
         \hline
         10 & 10 & 3.162 \\
         \hline
         0 & 1 & 1 \\
         \hline
    \end{tabular}
    \caption{Exemplo de uma escala mostrando o rácio de potência $x$, os rácios de amplitude $\sqrt{x}$ os dB equivalentes $10 \log_{10} x$.}
    \label{tab:tabela_racio}
\end{table}