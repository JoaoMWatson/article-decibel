\documentclass[
	article,			
	12pt,				
	oneside,			
	a4paper,						
	brazil,				
	sumario=tradicional
	]{abntex2}

\input{./config/packages}

\renewcommand{\backrefpagesname}{Citado na(s) página(s):~}

\renewcommand{\backref}{}

\renewcommand*{\backrefalt}[4]{
	\ifcase #1 %
		Nenhuma citação no texto.%
	\or
		Citado na página #2.%
	\else
		Citado #1 vezes nas páginas #2.%
	\fi}%

\titulo{Bel e Decibel}
\tituloestrangeiro{}

\autor{
    \Large Centro Universitário Senac \\
    Bacharelado em Ciência da Computação 
\\[0.75cm]
\normalsize João Pedro Martins Watson\thanks{Estudante em Bacharelado de Ciência da Computação}
}

\local{São Paulo}
\data{São Paulo \\ 7 de Junho 2022}

\definecolor{blue}{RGB}{41,5,195}

\makeatletter
\let\@fnsymbol\@arabic
\hypersetup{
        %pagebackref=true,
		pdftitle={\@title}, 
		pdfauthor={\@author},
        pdfsubject={Bel e Decibel},
        pdfcreator={LaTeX with abnTeX2},
		pdfkeywords={Bel}{Decibel}{Som}, 
		colorlinks=true,       		% false: boxed links; true: colored links
        linkcolor=blue,          	% color of internal links
        citecolor=blue,        		% color of links to bibliography
        filecolor=magenta,      	% color of file links
		urlcolor=blue,
		bookmarksdepth=4
}
\makeatother

\makeindex

\setlrmarginsandblock{3cm}{3cm}{*}
\setulmarginsandblock{3cm}{3cm}{*}
\checkandfixthelayout

% O tamanho do parágrafo é dado por:
\setlength{\parindent}{1.5cm}

\setlength{\parskip}{0.2cm}

\SingleSpacing

\begin{document}

\selectlanguage{brazil}
\frenchspacing 
\maketitle

\begin{resumoumacoluna}
    
	Este artigo trata-se de uma pesquisa e aprofundamento sobre os conceitos e definições de Bel(B) e Decibel(dB).
	Contando sua origem, fundamentos, utilidade e destrinchando os cálculos e a matemática que os definem. 
	A principal função deste trabalho será entender o por que, e, para que, esses conceitos são importantes 
	e por que estudá-los.
    
    \vspace{\onelineskip}
    
    \noindent
    \textbf{Palavras-chave}: Bem. Decibel. Som.
\end{resumoumacoluna}

\textual

\newpage

\section{Introdução}

O decibel (dB) é uma unidade de medida relativa, adimensional, 
correspondente à décima parte de um bel (B), que expressa o rácio de uma grandeza 
física (geralmente energia ou intensidade) em relação a um nível de referência 
specificado ou implícito, expressa numa escala logarítmica de base 10 ($\log_{10}$). 
Um valor em dB expressa a relação entre dois valores de um nível de potência ou de um 
nível de campo (ou nível de potência-raiz) numa escala logarítmica. 
Dois sinais cujos níveis difiram em 1,0 dB têm uma relação 
de potência de 101/10 (aproximadamente 1,26) ou uma relação 
de potência-raiz de 101⁄20 (aproximadamente 1,12).

\section{Discussão}

\subsection{Definição}
A unidade bel é usada para caracterizar o logaritmo decimal da razão entre duas quantidades similares de energia ou potência $P_1$ e $P_2$:\\
    $$Q_(P) = \log\frac{P_1}{P_2} B = 10 \log\frac{P_1}{P_2} dB$$\\
Para $Q_{(P)}$, por exemplo, o valor 1 B resulta quando a razão de potência é $P_1/P_2 = 10$.
$$1dB = \frac{1}{10} B$$

\section{Tabela}
\begin{table}[hb]
    \centering
    \begin{tabular}{|c|c|c|}
         \hline
         dB & Rácio de Potências & Rácio de Amplitude \\
         \hline
         100 & 10 000 000 000 & 100 000 \\
         \hline
         90 & 1 000 000 000 & 31 623 \\
         \hline
         80 & 100 000 000 & 10 000 \\
         \hline
         70 & 10 000 000 & 3 162 \\
         \hline
         60 & 1 000 000 & 1 000 \\
         \hline
         50 & 100 000 & 316.2 \\
         \hline
         40 & 10 000 & 100 \\
         \hline
         30 & 1 000 & 31.62 \\
         \hline
         20 & 100 & 10 \\
         \hline
         10 & 10 & 3.162 \\
         \hline
         0 & 1 & 1 \\
         \hline
    \end{tabular}
    \caption{Exemplo de uma escala mostrando o rácio de potência $x$, os rácios de amplitude $\sqrt{x}$ os dB equivalentes $10 \log_{10} x$.}
    \label{tab:tabela_racio}
\end{table}

\newpage
\input{pages/bibliografia}

\end{document}
